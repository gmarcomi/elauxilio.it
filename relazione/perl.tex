\subsection{Perl}

Gli script in perl/cgi si occupano essenzialmente del templating e dell'estrazione, manipolazione e popolamento dei dati da visualizzare.\\
Il codice dei moduli è stato realizzato seguendo il paradigma della programmazione ad oggetti; in questo progetto sono stati utilizzati alcuni pattern al fine di rendere l'architettura generale più chiara e di rendere il codice più mantenibile e riusabile. L'architettura di sistema utilizzata scelta è stata l'architettura model view presenter, adattata alla realizzazione di un sito web, mentre è stato utilizzato il pattern Composite per aggiungere funzionalità ad un oggetto.\\
Per mia decisione, l'intero sito si appoggia sui vari script cgi rendendolo totalmente dinamico, infatti index.html reindirizza direttamente alla pagina index.cgi.
Questa scelta è stata presa per permettere la visualizzazione di una casella utente all'interno della barra di navigazione, visualizzando il nome dell'utente e il link alla area utente.\\

I file perl/cgi/tmpl sono strutturati nel seguente modo:

\begin{enumerate}
  \item \texttt{model}: la cartella contiene tutti gli oggetti necessari alla rappresentazione della logica di business all'interno del sito, quindi la parte deputata al reperimento, elaborazione e modifica dei vari file xml.
  \item \texttt{presenter}: il package contiene tutti gli oggetti necessari per la presentazione del sito(script e template);
  \item \texttt{changePassword.pl}: script che si occupa di effettuare le corrette operazioni allo scopo di modificare la password di un utente, ricevendo dati dal form contenuto nella pagina cambio-password.cgi;
  \item \texttt{logout.pl}: script  che si occupa di disautenticare correttamente l'utente dal sito;
  \item \texttt{manageAnswers.pl}: si occupa di effettuare le operazione di inserimento e cancellazione dei vari commenti, questo script riceve le varie richieste di tipo post dalle varie pagine notizie del sito;
  \item i vari script cgi delle pagine del sito, i quali utilizzano gli oggetti contenuti nei vari package, oltre ai moduli moduli CPAN. Ogni script che rappresentano pagine in cui è necessario essere autenticati prima di accedervi, effettuano un controllo della sessione, se non presente reindirizzando nella home.
\end{enumerate}

\subsubsection{Package model}
Il package model contiene:
\begin{itemize}
  \item \texttt{auxils}: package contenenti gli oggetti necessari alla gestione dei dati contenuti nel file auxils.xml;
  \item \texttt{user}: package contenenti gli oggetti necessari alla gestione dei dati contenuti nel file users.xml;
  \item \texttt{ComFunc.pm}: libreria contenenti funzioni utilizzati da più file.
\end{itemize}

Il sottopackage auxils è composto da:
\begin{itemize}
  \item \texttt{Answers.pm}: oggetto utilizzato per la gestione delle risposte all'interno delle varie notizie(auxils);
  \item \texttt{Auxil.pm}: oggetto utilizzato per la gestione delle informazioni dei vari auxils pubblicati(notizie).
\end{itemize}

Mentre il sotto package user è composto da 
\begin{itemize}
  \item \texttt{User.pm}: oggetto contenente metodi di tipo query per il file users.xml;
  \item \texttt{Register.pm}: oggetto utilizzato per la corretta registrazione di un utente all'interno del sito;
  \item \texttt{ModifyUser.pm}: oggetto utilizzato per la modifica di un utente all'interno di un sistema.
\end{itemize}

\subsubsection{Package presenter}
Il package presenter è cosi strutturato:
\begin{itemize}
  \item \texttt{view}: questo package contiene i vari template. A sua volta è suddiviso in common e pages, common contiene i template innestati in altri template, mentre pages contiene i template delle varie pagine del sito;
  \item \texttt{Presenter.pm}: oggetto utilizzato per la composizione e renderizzazione dei vari template utilizzati da una pagina;
  \item \texttt{PresenterAnswers.pm}: oggetto container per l'oggetto Presenter, necessario alla corretta presentazione dei commenti all'interno di una notizia.
\end{itemize}