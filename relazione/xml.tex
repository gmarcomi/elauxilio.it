\subsection{XML}
I motivi per cui è stato preferito XMLSchema a DTD, come linguaggio per descrivere la struttura dei file XML, va a ricercarsi nella maggiore espressività di XSD rispetto a quella di DTD, la possibilità di definire tipo e la possibilità di utilizzare nativamente namespace. Non sono stati utilizzati namespace in quanto ritenuti non assolutamente necessari; decisione motivata dalla non presenza di schemi diversi da quelli definiti, come schemi esterni.\\
Il sito possiede due file xml: users.xml e auxils.xml.
Il primo file contiene le varie informazioni degli utenti di Auxils.it, mentre auxils.xml contiene le varie informazioni delle notizie all'interno del sito, informazioni come titolo, descrizione, autore, data di creazione e commenti degli utenti.\\
Tutti i file xml e il relativo schema sono stati testati mediante W3C XML Schema (XSD) Validation online.

\subsubsection{Attributi}

Gli attributi sono stati usati solamente per discriminare gli utenti admin da quelli normali.
Alcune caratteristiche inizialmente pensate come attributi, come topic, sono state definite mediante l'uso di elementi, in quanto è sicuramente più estendibile e offre maggiori possibilità di definizione di un tipo.

\subsubsection{Modello adottato}
Il modello adottato è stato quello delle Tende veneziane, in quanto sono stati considerati flessibilità di namespacing e riusabilità fattori di primaria importanza.\\
