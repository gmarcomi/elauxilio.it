\section{Abstract}
La seguente relazione ha il compito di illustrare le fasi di progettazione, realizzazione e test del progetto da me svolto.
Il progetto consiste nella realizzazione di un sito che sia al contempo usabile ed accessibile da tutte le categorie di utenti sia che siano disabili, sia che siano normodotati.\\
Il sito da me realizzato è un sito di tipo blog intitolato Auxils.it, il quale tratta notizie riguardanti il mondo open source.\\ 
Il sito dovrà possedere le seguenti caratteristiche:

\begin{enumerate}
  \item il sito web deve essere realizzato con lo standard XHTML Strict;
  \item il layout deve essere realizzato con CSS puri;
  \item il sito web deve rispettare la completa separazione tra contenuto, presentazione e comportamento
  \item il sito web deve essere accessibile a tutte le categorie di utenti interessate;
  \item il sito web deve organizzare i propri contenuti in modo da poter essere facilmente reperiti da qualsiasi utente;
  \item le funzionalità introdotte attraverso l'uso di script Javascript devono essere disponibili in altri modi anche nel caso in cui Javascript sia stato disattivato dall'utente.
  \item il sito web deve contenere pagine che utilizzino gli script per collezionare e pubblicare dati inseriti dagli utenti;
  \item i dati inseriti dagli utenti devono essere salvati in file XML e deve essere fornito il relativo schema XMLSChema.
\end{enumerate}

Le caratteristiche sopra enunciate includono \href{http://docenti.math.unipd.it/gaggi/tecweb/progetto.html}{le caratteristiche richieste} dalla prof.ssa Gaggi affinché un progetto possa essere considerato di risultato almeno sufficiente.\\
La realizzazione di questo sito per il progetto didattico è stata motivata dal mio desiderio di realizzare un blog accessibile a qualsivoglia categoria di utente, come il concetto stesso dell'open source non discrimina varie categorie di utenti, stessa cosa dovrà fare il sito. Quindi l'idea di un sito che parla di open source e che sia al contempo accessibile e usabile è abbastanza valida e meritevole di essere concretizzata. \\
Al fine di garantire un buon livello di accessibilità, sono state seguire le linee guida del W3C.